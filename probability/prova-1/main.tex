\documentclass[a4paper, 11pt]{article}
\usepackage[utf8]{inputenc}
\usepackage[margin=1in]{geometry} 
\usepackage{amsmath,amsthm,amssymb}
\usepackage{listings}
\usepackage{graphicx}
\usepackage{indentfirst}
\renewcommand{\baselinestretch}{1.2}
%\setlength{\tabcolsep}{0.5em}
\renewcommand{\arraystretch}{1.2}
\usepackage{subcaption}
\usepackage{float}
\usepackage{dsfont} %change indicator function
\graphicspath{ {images/} }
\usepackage{comment} % enables the use of multi-line comments (\ifx \fi) 
\usepackage{fullpage} % changes the margin

\begin{document}
%Header-Make sure you update this information!!!!
\noindent
{\Large\textbf{Prova 1} \hfill \\
Probabilidade \hfill Primeiro Semestre\\
Wyara Vanesa Moura e Silva \hfill 2022\\}

\section*{Questão 1} Seja $\mathbf{X}$ uma variável aleatória com função de probabilidade de parâmetros $\theta > 0$ dada por 
\begin{equation*}
\begin{array}{lclll}
\mathds{P}(\mathbf{X} = k) & = & \dfrac{e^{-\theta}\theta^{k}}{4k!}(1 + \alpha k), \; & \; k=0,1,2,\ldots
\end{array}
\end{equation*}
\noindent
Se $\alpha$ é constante positiva.

1. Calcular o valor de $\alpha$.

\noindent
\textit{Solução:} 
\begin{equation*}
\begin{array}{rclll}
\displaystyle\sum_{k=0}^{\infty}\mathds{P}(\mathbf{X} = k) & = & 1 \\[15pt]

\displaystyle\sum_{k=0}^{\infty}\dfrac{e^{-\theta}\theta^{k}}{4k!} + \displaystyle\sum_{k=0}^{\infty}\alpha k \cdot \dfrac{e^{-\theta}\theta^{k}}{4k!} & = & 1 \\[10pt]

\dfrac{1}{4}\displaystyle\sum_{k=0}^{\infty}\dfrac{e^{-\theta}\theta^{k}}{k!} + \dfrac{\alpha}{4}\displaystyle\sum_{k=0}^{\infty} k \cdot \dfrac{e^{-\theta}\theta^{k}}{k!} & = & 1 \\[10pt]

\alpha \cdot \displaystyle\sum_{k=1}^{\infty}\cdot \dfrac{e^{-\theta}\theta^{k}}{(k-1)!} & = & 3 \\[10pt]

\alpha \cdot e^{-\theta} \cdot \theta \cdot \displaystyle\sum_{k=1}^{\infty}\cdot \dfrac{\theta^{k-1}}{(k-1)!} & = & 3 \\[10pt]

\alpha \cdot e^{-\theta} \cdot \theta \cdot \left\{ 1 + \dfrac{\theta}{1!} + \dfrac{\theta^{2}}{2!} + \dfrac{\theta^{3}}{2!} + \ldots\right\} & = & 3 \\[10pt]

\alpha \cdot e^{-\theta} \cdot \theta \cdot e^{\theta} & = & 3 \\
\alpha & = & \dfrac{3}{\theta} \\
\end{array}
\end{equation*}

2. Provar que podemos escrever
\begin{equation*}
\begin{array}{lclll}
\mathds{P}(\mathbf{X} = k) & = & \dfrac{1}{4}\mathds{P}(\mathbf{Y} = k) + \dfrac{3}{4}\mathds{P}(\mathbf{T} = k).
\end{array}
\end{equation*}

onde $\mathbf{Y}$ e $\mathbf{Z}$ tem distribuição de Poisson de parâmetros $\theta$ e ademais $\mathbf{T} = 1 +\mathbf{Z}$.

Pode usar que se $\mathbf{W}$ é Poisson de parâmetro $\theta$, então

\begin{equation*}
\begin{array}{lclll}
\mathds{P}(\mathbf{W} = k) & = & \dfrac{e^{-\theta}\theta^{k}}{k!}, \; & \; k=0,1,2,\ldots
\end{array}
\end{equation*}

\noindent
\textit{Solução:} \\
\begin{equation*}
\begin{array}{lclll}
\mathds{P}(\mathbf{X} = k) & = & \dfrac{1}{4}\cdot\dfrac{e^{-\theta}\theta^{k}}{k!} + \dfrac{3}{4}\cdot\dfrac{k}{\theta}\cdot\dfrac{e^{-\theta}\theta^{k}}{k!} \\[10pt]

& = & \dfrac{1}{4}\cdot\dfrac{e^{-\theta}\theta^{k}}{k!} + \dfrac{3}{4}\cdot\dfrac{e^{-\theta}\theta^{k-1}}{(k-1)!} \\

\end{array}
\end{equation*}

$\mathbf{Z} \sim$ Poisson $(\theta)$.

\begin{equation*}
\begin{array}{lclll}
\mathds{P}(\mathbf{Z} = k) & = & \dfrac{e^{-\theta}\theta^{k}}{k!} \\[20pt]

\mathds{P}(\mathbf{T} = k) & = & \mathds{P}(\mathbf{Z} + 1 = k) \; = \; \mathds{P}(\mathbf{Z} = k - 1) \; = \; \dfrac{e^{-\theta}\theta^{k-1}}{(k-1)!} \\[20pt]

\mathds{P}(\mathbf{X} = k) & = & \dfrac{1}{4}\cdot\mathds{P}(\mathbf{Y} = k) \; = \; \dfrac{3}{4}\cdot\mathds{P}(\mathbf{T} = k) \\

\end{array}
\end{equation*}

\noindent
mistura de poisson simples será uma poisson deslocada.

\section*{Questão 2} 

Seja $\mathbf{X}$ uma variável aleatória geométrica com parâmetro $p$.

\begin{equation*}
\begin{array}{lclll}
\mathds{P}(\mathbf{X} = k) & = & p(1-p)^{k-1}, & 0<p<1, & k=1,2,\ldots
\end{array}
\end{equation*}

\noindent
Encontrar

1. Uma fórmula para $\mathds{P}(\mathbf{X} = k|\mathbf{X}>a)$, com $k \in \mathds{N}$, $a\in \mathds{R}$ e $a>0$.

\noindent
\textit{Solução:} \\
\begin{equation*}
\begin{array}{rclll}
\displaystyle\sum_{k=0}^{\infty}\mathds{P}(\mathbf{X} = k) & = & \displaystyle\sum_{k=0}^{\infty}p(1-p)^{k-1} & = & 1 \\[15pt]

\displaystyle\sum_{k=0}^{\infty}(1-p)^{k-1} & = & \dfrac{1}{p} \\[15pt]

1 + (1-p) + (1-p)^{2} + (1-p)^{3} + \ldots & = & \dfrac{1}{p} \\[15pt]

\dfrac{1}{1-(1-p)} & = & \dfrac{1}{p} \\[15pt]

\end{array}
\end{equation*}

\begin{equation*}
\begin{array}{rclll}
\mathds{P}(\mathbf{X} = k|\mathbf{X}>a) & = & \dfrac{\mathds{P}(\mathbf{X} = k;\mathbf{X}>a)}{\mathds{P}(\mathbf{X}>a)} & = & \left\{
    \begin{array}{rrlc}
         0, & \mbox{se} & k < a \\
         \dfrac{\mathds{P}(\mathbf{X} = k)}{\mathds{P}(\mathbf{X}>a)}, & \mbox{se} & k>a
    \end{array}
\right.\\[15pt]

\end{array}
\end{equation*}

\noindent
assim,

\begin{equation*}
\begin{array}{rclll}
\mathds{P}(\mathbf{X} > a) & = & \displaystyle\sum_{k=\lfloor a \rfloor + 1}\mathds{P}(\mathbf{X} = k)  \\[15pt]

& = & \displaystyle\sum_{k=\lfloor a \rfloor + 1}p(1-p)^{k-1}  \\[15pt]

& = & p \left\{ \left[(1-p)^{\lfloor a \rfloor} + (1-p)^{\lfloor a \rfloor + 1} + (1-p)^{\lfloor a \rfloor + 2} + \ldots \right] \right\}  \\[15pt]

\end{array}
\end{equation*}

\begin{equation*}
\begin{array}{rclll}

& = & p (1-p)^{\lfloor a \rfloor} \left[(1-p)^{\lfloor a \rfloor} + (1-p) + (1-p)^{2} + \ldots \right] \\[15pt]

& = & p (1-p)^{\lfloor a \rfloor} \cdot \dfrac{1}{p} \\[15pt]

& = & (1-p)^{\lfloor a \rfloor} \\[15pt]
\end{array}
\end{equation*}

\noindent
portanto,

\begin{equation*}
\begin{array}{rclll}
\mathds{P}(\mathbf{X} = k|\mathbf{X}>a) & = & \left\{
    \begin{array}{rrlc}
         0, & \mbox{se} & k < a \\[10pt]
         \dfrac{p(1-p)^{k-1}}{(1-p)^{\lfloor a \rfloor}}, & \mbox{se} & k>a
    \end{array}
\right.\\[25pt]

\end{array}
\end{equation*}

2. O valor de $\mathds{P}(\mathbf{X} = 10|\mathbf{X}>2\pi)$, para $p=1/2$.

\noindent
\textit{Solução:} \\

$p=1$, $a=2\pi$ e $k=10$

\begin{equation*}
\begin{array}{rclllll}
\mathds{P}(\mathbf{X} = 10|\mathbf{X}>2\pi) & = & \left(  \dfrac{1}{2}\right)\cdot\left(  \dfrac{1}{2}\right)^{10-1-6} & = & \left(  \dfrac{1}{2}\right)^{4} & = & \dfrac{1}{16}

\end{array}
\end{equation*}


\section*{Questão 3} 

Seja $\mathbf{X}$ variável aleatória exponencial de parâmetro $\theta > 0$ e $m$ uma constante positiva, então

\begin{equation*}
\begin{array}{lclll}
f_{X}(x) & = & \theta e^{-\theta x}, & x \geq 0.
\end{array}
\end{equation*}

\noindent
Definimos
\begin{equation*}
\begin{array}{lclll}
\mathbf{Z} := \mbox{min}\{\mathbf{X},m\} & = & \mathbf{X} \mathds{1}(\mathbf{X}\leq m) + m \mathds{1}(\mathbf{X} > m), & m>0
\end{array}
\end{equation*}

\noindent
Onde $\mathds{1}(.)$ é a função indicadora.

1. Encontrar a função de distribuição de $\mathbf{Z}$, é uma mistura?

\noindent
\textit{Solução:} \\
\begin{equation*}
\begin{array}{lcllllll}
\mathbf{Z} & = & \left\{
    \begin{array}{rrlc}
         \mathbf{X}, & \mbox{se} & \mathbf{X} \leq m \\
         m, & \mbox{se} & \mathbf{X} > m
    \end{array}
\right. \\[15pt]
\end{array}
\end{equation*}

\begin{equation*}
\begin{array}{rclll}
\mathds{P}(\mathbf{Z} = m) & = & \mathds{P}(\mathbf{X} > m) & = & e^{-\theta m}\\[15pt]

F_{\mathbf{Z}}(z) & = & \mathds{P}(\mbox{min}\{ \mathbf{X}; m\} \leq z ) \\[15pt]

& = & 1 - \mathds{P}(\mbox{min}\{ \mathbf{X}; m\} > z ) & = & 1 - [\mathds{P}(\mathbf{X} > z, m>z)] \\[15pt]

\end{array}
\end{equation*}

$\mathds{P}(\mathbf{Y}=m) = 1$: variável degenerada (constante), $\mathbf{X}$ e $\mathbf{Y}$ são independentes.

\begin{equation*}
\begin{array}{rclll}
& = & 1 - [\mathds{P}(\mathbf{X} > z, \mathbf{Y}>z)] & = & 1 - [\mathds{P}(\mathbf{X} > z)\cdot\mathds{P}(\mathbf{Y}>z)] \\[15pt]

F_{\mathbf{Z}}(z) & = & 1 - [\mathds{P}(\mathbf{X} > z)\cdot\mathds{P}(\mathbf{Y}>z)] \\[15pt]

\end{array}
\end{equation*}
 
\begin{equation*}
\begin{array}{lcllllll}
\mathds{P}(\mathbf{Y} > z) & = & \mathds{P}(m > z) & = & \left\{
    \begin{array}{rrlc}
         1 & \mbox{se} & m > z \\
         0 & \mbox{se} & m \leq z
    \end{array}
\right. \\[15pt]
\end{array}
\end{equation*}

Assim,
\begin{equation*}
\begin{array}{rclll}
F_{\mathbf{Z}}(z) & = & 1 - e^{-\theta z}\cdot\mathds{1}(z < m)  \\[15pt]

\end{array}
\end{equation*}
\noindent
sim, é uma mistura.\\
 
2. Se for uma mistura, indicar as componentes da mistura. Fazer um desenho da função de distribuição de $\mathbf{Z}$.

\noindent
\textit{Solução:} \\


\section*{Questão 4}

Dada a variável aleatória $\mathbf{X}$ e a constante $a>0$, com densidade 

\begin{equation*}
\begin{array}{lcllllll}
f_{\mathbf{X}}(x) & = & \left\{
    \begin{array}{rrlc}
         a/2, & \mbox{se} & -1 < x \leq 0\\
         \frac{a}{2}e^{-x}, & \mbox{se} & x>0
    \end{array}
\right. \\
\end{array}
\end{equation*}

1. Encontrar o valor da constante $a$.

\noindent
\textit{Solução:} 

\begin{equation*}
\begin{array}{rclll}
\dfrac{a}{2} + \displaystyle\int_{0}^{\infty}\dfrac{a}{2}e^{-x}dx = 1 \\[15pt]
\dfrac{a}{2} + \dfrac{a}{2}\displaystyle\int_{0}^{\infty}e^{-x}dx = 1 \\[15pt]
a = 1 \\[15pt]
\end{array}
\end{equation*}

2. Encontrar a função de distribuição e função de densidade da variável aleatória $\mathbf{Y}=\mathbf{X}^{2}$.

\noindent
\textit{Solução:} \\
\begin{equation*}
\begin{array}{rclll}
F_{\mathbf{Y}}(y) = F_{\mathbf{X}}(\sqrt{y}) - F_{\mathbf{X}}(-\sqrt{y}); & & y > 0. \\[15pt]
\end{array}
\end{equation*}

\begin{equation*}
\begin{array}{rclll}
f_{\mathbf{Y}}(y) = \dfrac{1}{2\sqrt{y}} \left[ f_{\mathbf{Y}}(\sqrt{y}) + f_{\mathbf{Y}}(-\sqrt{y})\right]; & & y > 0. \\[15pt]
\end{array}
\end{equation*}

\begin{equation*}
\begin{array}{rclll}
0 < y < 1; & & f_{\mathbf{Y}}(y) & = & \dfrac{1}{2\sqrt{y}}\cdot \left[ \dfrac{1}{2} e^{-\sqrt{y}} + \dfrac{1}{2} \right] \\[15pt]

y \geq 1; & & f_{\mathbf{Y}}(y) & = & \dfrac{1}{2\sqrt{y}}\cdot \dfrac{1}{2} e^{-\sqrt{y}} + 0 \\[15pt]
\end{array}
\end{equation*}

\end{document}
